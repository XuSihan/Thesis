% !TeX root = ../main.tex
% -*- coding: utf-8 -*-


\chapter{基于梯度上升决策树的函数抽取重构推荐} 
\label{chpt:relatedwork}

\section{引言}
\section{问题描述}

\begin{table}[!t]
  \renewcommand{\arraystretch}{1.3}
  % if using array.sty, it might be a good idea to tweak the value of
  % \extrarowheight as needed to properly center the text within the cells
  \caption{Method-level Metrics of the Candidates in the Example}
  \label{example_metrics}
  \centering
  %% Some packages, such as MDW tools, offer better commands for making tables
  %% than the plain LaTeX2e tabular which is used here.
  %\begin{tabular}{|p{0.1cm}p{0.55cm}p{0.6cm}p{0.6cm}p{0.8cm}||p{0.1cm}p{0.55cm}p{0.6cm}p{0.6cm}p{0.8cm}|}
  %\begin{tabular}{|p{0.1cm}p{0.55cm}p{0.6cm}p{0.6cm}p{0.8cm}||p{0.1cm}p{0.55cm}p{0.6cm}p{0.6cm}p{0.8cm}|}
  \begin{tabular}{ccccc|ccccc}
  \toprule No. &Cand &M1 &M2 &M3 &No. &Cand &M1 &M2 &M3\\ \midrule 1 &2-7 &8 &0 &6
   &9 &3-10 &6 &0 &6\\
   2 &2-8 &10 &0 &6 &10 &3-11 &13 &0 &6\\
   3 &\textbf{2-9} &11 &0 &6 &11 &3-12 &21 &0 &6\\
   4 &2-10 &13 &0 &6 &12 &4-9 &0 &0 &3\\
   5 &2-11 &28 &3 &6 &13 &4-10 &0 &0 &3\\
   6 &2-12 &36 &10 &6 &14 &5-7 &0 &0 &3\\
   7 &3-8 &4 &0 &6 &15 &5-8 &0 &0 &3\\
   8 &3-9 &5 &0 &6 &16 &5-9 &0 &0 &4\\
   \bottomrule
  \end{tabular}
  \end{table}
  


\begin{figure*}
  \centering
    \begin{lstlisting}[columns=fullflexible,label={lst:label}]
   public String calculateSignature(...) {
          String baseUrl = baseUrl(uri);
          String encodedParams = encodedParams(oauthTimestamp, nonce, formParams, queryParams);
          StringBuilder sb = StringUtils.stringBuilder();
          sb.append(method);
          sb.append('&');
          Utf8UrlEncoder.encodeAndAppendQueryElement(sb, baseUrl);
          sb.append('&');
          Utf8UrlEncoder.encodeAndAppendQueryElement(sb, encodedParams);
          ByteBuffer rawBase = StringUtils.charSequence2ByteBuffer(sb, UTF_8);
          byte[] rawSignature = mac.digest(rawBase);
          return Base64.encode(rawSignature);
      }
    \end{lstlisting}
    \caption{An Example Method from AsyncHttpClient Library}
    \label{marker}
  \end{figure*}
\section{研究方法}
\subsection{特征提取算法}

\begin{figure}[bth]
  {\scriptsize
	\begin{center}
	  \begin{algorithmic} [1]
%                        \Procedure{feature\_extract}{$m,b,P$}
			\REQUIRE  : The input method \textit{m}, the selected block
			\textit{b} and the set of program elements \textit{P}
			\ENSURE The feature vector % \textit{feature\_extract(\textit{m},\textit{b},\textit{P})
			\STATE $S_M  \leftarrow getStmts(m)$
			\STATE $S_B  \leftarrow getStmts(b)$
			\FOR{$p$ in $P$}
				\STATE $E \leftarrow getElmts(b, p)$
				\FOR{$e$ in $E$}			
					\STATE $freq_M \leftarrow getFreq(m, e)$
					\STATE $freq_B \leftarrow getFreq(b, e)$
					\STATE $ratio_e \leftarrow freq_B / freq_M$
				\ENDFOR
				\STATE $idx \leftarrow \textit{idxOfMax}(ratio_E)$
				\STATE $cp \leftarrow ratio_{idx}$
				\STATE $loc \leftarrow getLOC(b)$				
				\STATE $count \leftarrow 0$				
				\FOR{$s$ in $S_B$}
					\IF{$E_{idx}$ in $getElmts(s)$}
						\STATE $count \leftarrow count + 1$
					\ENDIF
				\ENDFOR
				\STATE $ch \leftarrow count / loc$ 
				\STATE $setFeature(p, cp, ch)$
			        \ENDFOR
%                                \EndProcedure
		\end{algorithmic}
	\end{center}
        }
	\caption{Extraction Algorithm of Metric-related Features}
	\label{feature_algorithm}
\end{figure}
\subsection{梯度上升决策树模型}
\subsection{候选函数抽取重构操作生成}
\section{实验设置}
\subsection{数据集和对比实验}
\begin{table}[!t]
  \renewcommand{\arraystretch}{1.3}
  % if using array.sty, it might be a good idea to tweak the value of
  % \extrarowheight as needed to properly center the text within the cells
  \caption{Statistics of Subject Systems}
  \label{benchmark}
  \centering
  \begin{tabular}{ccc}
  \toprule 
  Projects &Methods &Refactor.\\ \midrule
  JHotDraw &56 &56 \\ 
  Junit &25 &25 \\ 
  MyWebMarket &23 &35 \\ 
  SelfPlanner &12 &13 \\ 
  Wikidev &14 &26 \\ \midrule
  Total &130 &155 \\ 
  \bottomrule
  \end{tabular}
  \end{table}
\subsection{评价指标}
\section{实验结果}

\begin{table}[!t]
  \renewcommand{\arraystretch}{1.3}
  % if using array.sty, it might be a good idea to tweak the value of
  % \extrarowheight as needed to properly center the text within the cells
  \caption{Accuracy of Different Approaches}
  \label{accuracy}
  \centering
  \begin{tabular}{cc|ccc}
  \toprule
   Tools &Tolerance &Precision &Recall &F-measure\\ 
  \hline
  \multirow{4}{*}{GEMS$^{GB}$}&$1\%$ &\bf{22.5\%} &\bf{54.2\%} &\bf{31.8\%} \\ 
  &$2\%$ &\bf{28.5\%} &\bf{59.8\%} &\bf{38.6\%} \\ 
  &$3\%$ &\bf{34.3\%} &\bf{62.6\%} &\bf{44.3\%} \\ 
  \hline
  \multirow{4}{*}{JExtract}&$1\%$ &12.6\% &52.2\% &20.4\% \\ 
  &$2\%$ &13.1\% &\bf{59.3\%} &21.5\% \\ 
  &$3\%$ &15.0\% &61.9\% &24.2\% \\ 
  \hline
  \multirow{4}{*}{SEMI}&$1\%$ &12.9\% &38.0\% &19.2\% \\ 
  &$2\%$ &14.6\% &47.0\% &22.3\% \\ 
  &$3\%$ &18.8\% &55.5\% &28.1\% \\ 
  \hline
  \multirow{4}{*}{JDeodorant}&$1\%$ &17.4\% &14.8\% &16.0\% \\
  &$2\%$ &21.1\% &18.4\% &19.7\% \\ 
  &$3\%$ &28.0\% &23.8\% &25.7\% \\ 
  \bottomrule
  \end{tabular}
  \end{table}



  
\begin{table}[!t]
  \renewcommand{\arraystretch}{1.3}
  % if using array.sty, it might be a good idea to tweak the value of
  % \extrarowheight as needed to properly center the text within the cells
  \caption{Accuracy in Long methods}
  \label{long_methods}
  \centering
  \begin{tabular}{cc|ccc}
  \toprule
   Tools &Tolerance &Precision &Recall &F-measure\\ 
  \hline
  \multirow{4}{*}{GEMS$^{GB}$}&$1\%$ &13.3\% &31.9\% &18.8\% \\ 
  &$2\%$ &\bf{17.4\%} &\bf{41.5\%} &\bf{24.5\%} \\ 
  &$3\%$ &\bf{25.3\%} &\bf{46.2\%} &\bf{32.7\%} \\ 
  \hline
  \multirow{4}{*}{JExtract}&$1\%$ &6.6\% &16.1\% &9.4\% \\ 
  &$2\%$ &8.0\% &19.3\% &11.3\% \\ 
  &$3\%$ &8.0\% &19.3\% &11.3\% \\ 
  \hline
  \multirow{4}{*}{SEMI}&$1\%$ &\bf{16.4\%} &\bf{38.7\%} &\bf{23.0\%} \\ 
  &$2\%$ &\bf{17.9\%} &\bf{41.9\%} &\bf{25.0\%} \\ 
  &$3\%$ &19.1\% &45.1\% &26.9\% \\ 
  \hline
  \multirow{4}{*}{JDeodorant}&$1\%$ &12.0\% &9.6\% &10.7\% \\
  &$2\%$ &14.3\% &12.9\% &13.5\% \\ 
  &$3\%$ &16.0\% &12.9\% &14.2\% \\ 
  \bottomrule
  \end{tabular}
  \end{table}

\section{讨论}
\section{本章小结}




模板仅在 TeXLive 2016,TeXLive 2018 下测试通过。对于其它 TeX 发行版可能需要做个别修改。


本模板可以使用以下方式编译:
\begin{enumerate}
 \item \XeLaTeX [推荐]
\end{enumerate}

例如,
\begin{verbatim}
         xelatex main
         biber main      % 处理参考文献
         xelatex main   % 连续编译两遍以生成正确的文献引用。
\end{verbatim}





本模板用到 宋体、楷体、仿宋、黑体四种字体. 若需重新配置字体, 请修改 NKTfonts.cfg.
对于 Linux/Mac 下的 TeX Live 2009, 可能需要设置环境变量 OSFONTDIR, 具体内容请参考 texmf.cnf.


我们建议您使用\XeLaTeX\ 编译。与前两种方法相比,\XeLaTeX\  编译长文档的速度更快,
编译一篇一百多页的论文只需几秒的时间(SL9400 @ 1.86GHz)。

在改变编译方式前应先删除 *.toc 和 *.aux 文件,
因为不同编译方式产生的辅助文件格式可能并不相同。



注意:使用 \XeLaTeX\ 编译时,\XeTeX\ 的版本应不低于 0.9995.0(MiKTeX 2.8 或者 TeXLive 2009)。


\label{sec:ex:A}

引用章节号请参考如下格式: \ref{chpt:relatedwork}\ref{sec:ex:A}.



使用 \XeLaTeX\ 编译时,会自动在中英文转换时添加必要的空格。 使用 [PDF]\LaTeX\
编译时仅忽略中文之间的空格,而中英文之间的空格予以保留。
因此,不管何种编译方式,您都不需要在中英文间添加 $\tilde{}$ 以获得额外的空格。例如,

这是 English 中文 $x=y$ 测试

这是English中文$x=y$测试

可以看出,以上两行用 \XeLaTeX\ 编译的结果是相同的。



NKThesis 已经调用以下宏包,您无须重新调用。

\begin{center}
\tablecaption{NKThesis 预调用的宏包}
\begin{tabular}{l|l}
\hline
编译方式 & 调用的宏包\\ \hline
\XeLaTeX & xeCJK, CJKnumb, graphicx, mathptmx, amssymb,mathpazo,pgf,tikz \\ \hline
\end{tabular}
\end{center}


\label{sec:relatedwork:table}




一般情况下, 您不需要显式地设置字体. 如果确实需要, 请使用以下命令

\begin{verbatim}
宋体:  \rmfamily\upshape 或 \songti
黑体:  \bfseries 或 \heiti
楷体:  \itshape  或 \kaiti
仿宋:  \ttfamily 或 \fangsong
加粗:  \jiacu
\end{verbatim}


参考文献引用:
\cite{ChenCheChen2001,Nadkarni-1992,Hua-Wang-1973}
\cite{ZhuKeZhen,Huo,Example}\cite{JiangXiZhou,Timoshenko,Zhang-Wang,Ding,GB6447-86}
\cite[Theorem 2.1]{ZhuKeZhen}


本模板采用 biblatex 宏包管理参考文献。如果你对此不熟悉,可以
\begin{enumerate}
\item 参考宏包使用说明,或者
\item 手工排版参考文献,然后参考 nkthesis.bib 最后 3 条的格式录入。
\end{enumerate}








\begin{proof}[定理~\ref{thm:latex}的证明]
显然是错的.
\end{proof}

单个带编号的表达式
\begin{equation}\label{eq:a1}
x=y+z
\end{equation}

单个不带编号的表达式
\[
y=x-z.
\]

不带编号的多行表达式
\begin{eqnarray*}
x&=&y+z \\
 &=&z-s\\
 &<& 3. \\
 && \mbox{一些注释}
\end{eqnarray*}

带编号的多行表达式
\begin{eqnarray}
 x&=& y-z, \label{eq:aa1}\\
 y&=& x+z, \nonumber \\
 z&=&y-x. \label{eq:aa2}
\end{eqnarray}



引用:   定理\ref{thm:latex}的推论是什么呢?
方程式编号:  由(\ref{eq:a1})(\ref{eq:aa2})式.


环境 enumerate 已经被改写,增加了一个可选参数[字符串], 用以控制所进。例如,
\begin{verbatim}
  \begin{enumerate}
    \item This is an example.
    \item This is an example.
      \begin{enumerate}
      \item This is an example.
      \item This is an example.
    \end{enumerate}
  \end{enumerate}
  \begin{enumerate}[Mn]% 字符串"Mn"的宽度为增加的缩进。
                       % 缺省值为 [M]
    \item This is an example.
    \item This is an example.
      \begin{enumerate}[Mnn]% 字符串"Mnn"的宽度为增加的缩进。
      \item This is an example.
      \item This is an example.
    \end{enumerate}
  \end{enumerate}
\end{verbatim}
的输出为
  \begin{enumerate}
    \item This is an example.
    \item This is an example.
      \begin{enumerate}
      \item This is an example.
      \item This is an example.
    \end{enumerate}
  \end{enumerate}
  \begin{enumerate}[Mn]% 字符串"Mn"的宽度为增加的所进。
                       % 缺省值为 [M]
    \item This is an example.
    \item This is an example.
      \begin{enumerate}[Mnn]% 字符串"Mnn"的宽度为增加的所进。
      \item This is an example.
      \item This is an example.
    \end{enumerate}
  \end{enumerate}
