% !TeX root = ../main.tex
% -*- coding: utf-8 -*-

\chapter{基于层次注意力网络的函数名推荐}
\label{chpt:method}
\section{引言}
\section{问题描述}
\section{研究方法}
\subsection{层次序列表示}
\subsection{层次注意力网络模型}
\subsection{集束搜索}
\section{实验设置}
\subsection{数据集和对比实验}
\subsection{评价指标}
\section{实验结果}
\section{讨论}
\section{本章小结}




\begin{figure}
    \centering
    % !TeX root = ../main.tex

\begin{tikzpicture}[node distance=2cm]
 %定义流程图具体形状
 \tikzstyle{startstop} = [rectangle, rounded corners, minimum width=3cm, minimum height=1cm,text centered, draw=black, fill=red!30]
 \tikzstyle{io} = [trapezium, trapezium left angle=70, trapezium right angle=110, minimum width=3cm, minimum height=1cm, text centered, draw=black, fill=blue!30]
 \tikzstyle{process} = [rectangle, minimum width=3cm, minimum height=1cm, text centered, draw=black, fill=orange!30]
 \tikzstyle{decision} = [diamond, minimum width=3cm, minimum height=1cm, text centered, draw=black, fill=green!30]
 \tikzstyle{arrow} = [thick,->,>=stealth]
 
\node (start) [startstop] {Start};
\node (in1) [io, below of=start] {Input};
\node (pro1) [process, below of=in1] {Process 1};
\node (dec1) [decision, below of=pro1, yshift=-0.5cm] {Decision 1};
\node (pro2a) [process, below of=dec1, yshift=-0.5cm] {Process 2a};
\node (pro2b) [process, right of=dec1, xshift=2cm] {Process 2b};
\node (out1) [io, below of=pro2a] {Output};
\node (stop) [startstop, below of=out1] {Stop};
 
 %连接具体形状
\draw [arrow](start) -- (in1);
\draw [arrow](in1) -- (pro1);
\draw [arrow](pro1) -- (dec1);
\draw [arrow](dec1) -- (pro2a);
\draw [arrow](dec1) -- (pro2b);
\draw [arrow](dec1) -- node[anchor=east] {yes} (pro2a);
\draw [arrow](dec1) -- node[anchor=south] {no} (pro2b);
\draw [arrow](pro2b) |- (pro1);
\draw [arrow](pro2a) -- (out1);
\draw [arrow](out1) -- (stop);
\end{tikzpicture}

    \caption{\label{fig:exmaple2} 示例流程图2}
\end{figure}


\label{sec:method:code}

python 代码可以直接使用\textbf{python}环境

\begin{python}[caption={斐波那契Python}]
def fibonacci(n):
    # Fibonacci number
    if n < 0:
        return False
    if n <= 1:
        return n
    return fibonacci(n-2) + fibonacci(n-1)
\end{python}

C/C++ 代码可以直接使用\textbf{cpp}环境

\begin{cpp}[caption={斐波那契C++}]
unsigned long Fibonacci(int n)
{
    // Fibonacci start from 0
    if (n <= 1) 
    {
        return n;
    }
    else 
    {
        return Fibonacci(n - 1) + Fibonacci(n - 2);
    }
}
\end{cpp}

其他代码,使用\textbf{lstlisting}指明 \textbf{language}即可,如matlab代码

\begin{lstlisting}[caption={Matlab代码},language=Matlab]
function a = factorial(n)
% return n!
    if n==0
        a=1;
    else
        a=n * factorial(n-1);
    end
\end{lstlisting}