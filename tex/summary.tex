% !TeX root = ../main.tex
% -*- coding: utf-8 -*-
\chapter{总结与展望}
本章对本文的主要研究内容、创新点以及所解决的问题进行了总结,并通过对未来研究方向
的展望,提出了关于研究工作的进一步的想法和建议。

\section{研究内容总结}
软件维护是贯穿软件生命周期的重要软件活动,占据了整个软件生命周期中的大部分时间。
随着用户需求的不断添加,软件系统的规模和复杂度也不断提高,导致软件维护的难度越来
越大。与此同时,软件系统通常需要快速迭代才能适应当代社会日新月异的改变与需求,为
软件维护的效率和性能带来了重大的挑战。

本文的研究内容主要针对软件维护中最重要的两个软件维护类型--纠错性软件维护和完善性
软件维护。纠错性软件维护通过尽可能早地发现软件系统中存在的缺陷,使得软件能够正确
地运行,是完善性软件维护的基础;同时,完善性软件维护通过改进软件设计、提高软件系
统的易读性、可靠性和可维护性,能够提高纠错性软件维护的效率,并减少新缺陷的引入。

本文的研究方法为通过数据驱动的方法,挖掘数据中的潜在规律作为软件维护问题的解决方
案。虽然目前很难制定出完整而严密的规则来完全自动化进行软件维护,但软件维护的过程
是存在规律的,并不是随机的、无章可循的。数据挖掘为这种存在潜在规律但很却难通过规
则完美解决的难题提供了一种新的解决方案。与此同时,随着科技的发展,开源社区和版本
控制工具的成熟也为数据挖掘提供了大量可用的数据。从数据的角度来理解软件维护的过
程,将软件维护问题转化为数据挖掘问题,能够在提高软件维护效率和性能的同时,减少软
件维护的成本。

为了提高纠错性和完善性软件维护的效率和性能,本文通过数据挖掘的方法完成了以下研究
工作:

第一,为了提高纠错性软件维护的效率,本文研究了基于覆盖分析的多缺陷定位问题。由于
开发过程中很难发现所有潜在的缺陷,软件系统中存留的缺陷可能导致软件行为失效。为了
提高纠错性软件维护的效率,研究者提出了很多缺陷定位方法来尽可能早地发现导致软件行
为失效的原因。其中,基于覆盖分析的缺陷定位方法由于只需要统计测试用例执行时的覆盖
信息,通过特定的公式估算代码的可疑度来定位缺陷,而不需要对源代码进行建模和分析,
因此计算成本较低,在面对大规模软件系统时适用性较强。然而,现有的基于覆盖分析的缺
陷定位方法大多假设被失效用例执行越多、成功用例执行越少的代码越可疑。这种假设虽然
在单缺陷程序中可能性较大,但面对多缺陷程序时,由于造成程序失效的原因不止一个,因
此即使是缺陷相关代码也可以不被失效用例所覆盖。由于很难区分由不同缺陷导致的失效用
例,导致现有的基于覆盖分析的缺陷定位方法在面对多缺陷程序时的定位效率下降。为了提
高多缺陷程序的定位效率,本文受特征选择启发,提出了缺陷相关统计量,为每个失效用例
寻找距离最近的失效用例,作为最有可能与其由相同缺陷触发的失效用例。通过计算特征对
样本类别的重要性,找到覆盖后容易导致测试用例执行失效的特征,将其作为可能的缺陷相
关代码,从而进行缺陷定位。


第二,为了提高完善性软件维护的效率,本文研究了关于函数抽取重构机会的识别和推荐。
随着对软件系统的不断修改,软件系统越来越偏离原来的设计,导致软件系统的质量越来越
低。为了提高软件质量,需要及时对软件系统进行重构。针对最常用的重构类型之一,函数
抽取软件重构,目前已经有很多研究者提出了自动识别和推荐重构机会的方法。大多数函数
抽取重构机会推荐方法通过软件质量度量来评价候选重构机会。虽然提高软件质量是软件重
构的主要原因,但与纠错性软件维护不同,完善性维护很难有统一的、明确的标准来量化维
护的效果。除此以外,考虑到函数抽取软件重构的原因具有多样性,根据软件质量度量进行
函数抽取机会推荐的结果往往只能得到满足某种特定软件质量度量的重构机会,因此效果通
常不甚理想。为了解决这个问题,通过挖掘开源软件仓库中的函数抽取重构实例,并在特征
提取算法中融合了复杂度、内聚度和耦合度三个与函数抽取重构相关的重要软件质量概念以
及多种程序元素,本文构建了关于函数抽取重构机会的概率模型,学习如何进行函数抽取重
构。除此以外,通过对特征重要性进行分析,探索了函数抽取重构的原因,从而进一步理解
了软件维护人员进行函数重构的意图。

第三,为了提高软件系统的易读性和维护效率,本文研究了函数重命名的问题。在软件维护
过程中,最耗费时间和精力成本的是阅读代码的过程。准确的函数名可以提高代码阅读的速
度,从而提高软件维护的效率;不准确的函数名通常导致理解和维护软件系统的难度提高,
甚至在某些情况下可能导致代码缺陷。尽管函数名对软件系统的易读性和可维护性具有较大
的影响,但寻找具有总结能力且易读性强的函数名十分困难。随着自然语言处理技术的发
展,很多研究者将程序语言作为一种特殊语言,使用自然语言处理技术将代码当作文本来学
习。尽管编程语言与自然语言之间存在一定的相似性,程序语言中蕴含着丰富的结构信息,
直接将代码当做普通文本进行学习,得到的上下文信息不够准确,容易导致模型效果欠佳。
为了利用代码结构中所蕴含的信息,本文通过控制流分析,将函数体拆分为由代码基本块组
成的序列,每个代码基本块表示一个函数的子功能,由词项序列进行表示。利用层次注意力
机制,分别学习词项对代码块和代码块对函数体的重要性,使得模型能够识别出对函数名预
测有积极意义的词项和代码块。通过这样的方式,学习输入代码段的分布式表示,以及函数
名在输入代码段的分布式表示上的条件概率,从而为给定代码段预测函数名。

\section{创新点与主要贡献}

第一,本文提出了基于相关统计量的缺陷定位方法,将多缺陷定位问题转化为数据挖掘中的
特征选择问题。为了提高基于覆盖分析的多缺陷定位,本文提出将测试用例作为样本,测试
用例的执行结果作为二分类标签,测试用例的执行覆盖信息作为样本特征,根据特征对样本
类别的重要性来计算覆盖特征相关代码的可疑度。具体来说,本文借鉴了Relief特征选择算
法中的相关统计量的概念,为每个测试用例寻找距离最近的同类和异类测试用例,距离同类
越近、距离异类越远的特征被认为越能够区分样本的类别,因此特征重要性越大。该方法能
够为每个失效用例找到最有可能与其由同一缺陷所触发的失效用例,从而有效避免了多缺陷
之间的相互影响。此外,由于根据特征选择算法只能得到对测试用例执行结果影响较大的特
征,无法得到特征的取值是如何影响测试用例的结果。因此,本文在此基础上提出了缺陷相
关统计量,使覆盖后容易导致样本类别为失效的覆盖特征的缺陷相关统计量较高,根据缺陷
相关统计量的由高至低推荐可疑代码。最后本文为评价该方法设计了两组对比实验,分别在
单缺陷和多缺陷程序上,与当前主流的基于覆盖分析的缺陷定位方法进行比较,在开源软件
程序上证明了基于相关统计量的缺陷定位方法的有效性。

第二,通过挖掘开源软件仓库中真实存在的函数抽取重构实例,构建关于函数抽取重构机会
的概率模型。为了提高函数抽取重构的效率,本文首次提出了基于机器学习的函数抽取重构
推荐方法,通过对比开源软件系统中相邻两个版本,收集函数抽取重构实例作为训练数据
集。在特征提取阶段,分别提取了结构和功能特征,融合了复杂度、内聚度和耦合度三个与
函数抽取相关的软件质量因素。与软件质量度量不同,本文的特征提取算法考虑了多种程序
元素,包括变量、函数调用、类型等。在训练阶段,通过这样的方式将函数抽取重构实例表
示为特征序列,使用梯度上升决策树模型进行训练。在推荐阶段,为给定函数体生成所有合
法的候选函数抽取重构机会后,通过训练好的模型为每个重构机会分配一个概率,帮按照概
率由高至低推荐给用户。除此以外,本文开发了基于Eclipse的插件,为选定函数推荐函数
抽取重构机会。最后,通过在开源软件系统上的对比实验,比较和分析了四种当前流行的函
数抽取重构机会推荐工具的性能,并比较了梯度上升决策树与其它模型在推荐函数抽取重构
机会任务上的优劣,实验结果证明了基于梯度上升决策树的函数抽取重构机会推荐模型的有
效性。

第三,通过挖掘开源软件系统中函数命名实例,学习函数名在代码片段上条件分布。为了提
高软件系统的易读性和可维护性,本文通过挖掘GitHub上广受欢迎的开源软件系统,学习函
数命名的内在规律,从而为给定代码片段推荐合适的函数名。具体来说,本文在编码-解码
模型的基本框架上提出了基于层次注意力的函数名推荐模型。在编码器中,根据代码的控制
流信息,将输入代码段拆分为由多个代码基本块组成的序列,每个代码基本块为一个词项序
列;利用层次注意力机制分别学习词项对代码块、代码块对代码片段的重要性,从而将输入
代码片段编码为一个分布式表示向量;在解码器中,使用基于门控循环单元的模型将编码器
的输出作为解码器的输入向量,依次输出词项序列,直到输出词项为结束符号为止。在函数
名推荐阶段,针对给定函数体,用训练好的模型进行预测,使用集束搜索生成具有概率的函
数名列表,并按照概率由高至低推荐给用户。最后,在10个开源软件系统上的对比实验证明
了基于层次注意力的函数名推荐模型的有效性,证明了本文方法能够在一定程度上学到代码
片段的语义。

\section{研究工作展望}
本文旨在通过数据挖掘提高软件维护关键技术的效率,针对软件维护中最主要的两个软件维
护类型,纠错行软件维护和完善性软件维护的效率,提出基于数据驱动的方法,将软件维护
问题转化为数据挖掘问题,从数据的角度理解软件维护的过程,构建相应的模型并实现了原
型系统。在一定程度上提高了纠错性软件维护和完善性软件维护的效率。然而,在实际的应
用过程中,本文方法仍然存在一定的局限性和可以改进的方向,主要体现在以下方面:

(1)基于数据驱动的方法对数据集质量的依赖性较大。由于基于数据驱动的方法通常假设
能够从数据中发现潜在的规律走位解决方案,因此数据集质量的好坏对模型的结果通常有较
大的影响。如果数据集中存在较多的噪音或者数据的分布较为不平衡,容易导致模型的性能
随之降低。例如在基于相关统计量的缺陷定位方法中,若测试用例集中只存在由某一种缺陷
引发的失效用例,而另一种缺陷没有或极少被触发,则很难通过该方法迅速定位到另一个缺
陷。为了避免这种情况,本文在实验中过滤了不能够被触发的软件缺陷,然而在实际使用
中,很难完全避免偶然触发性的软件缺陷。再例如在基于数据驱动的软件重构推荐中,若训
练数据集中夹杂着较多的噪音,则模型的性能很可能会受到影响而下降。


(2)虽然基于梯度上升决策树的函数抽取重构机会推荐模型,在一定程度上提高了函数抽
取重构的效率,但该方法通过特征提取的方式对函数抽取重构实例进行表示,而没有使用原
始特征,可能导致部分与软件重构相关的信息在特征提取阶段被丢弃,从而可能降低函数抽
取重构推荐按的准确性。

(3)本文针对软件重构中最常用的两个软件重构类型进行研究,但针对其它软件重构类
型,是否可能通过数据挖掘构建通用的软件重构机会识别和推荐模型。

针对上述本文方法的局限性,未来将进行更加深入的研究,例如:

(1)为了解决上述数据不平衡的情况,一种可能的方式是建立完善的数据评价体系,通过
对数据集进行抽样,根据抽样的结果估算数据集中各类别数据的分布情况;另一个可行的方
式是在使用可视化的方式观察数据集的分布。

(2)为了尽可能地减少数据集中噪音对模型的影响,接下来一方面可以通过完善数据集收
集的方式,增加数据集规模来减少噪音对模型的影响力;另一方面可以对通过模型加入一些
规则化,通过降低模型的复杂度,来减少模型对噪音数据的敏感性。

(3)关于对软件重构机会推荐的研究,本文接下来的研究方向是构建利用原始特征构建软
件重构概率模型。虽然对重构实例的表示是一个难题,但通过将其转化为对代码属性的预
测,例如在代码表示模型的基础上,通过预测在软件重构给定代码上的条件分布来推荐软件
重构机会。

