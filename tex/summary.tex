% !TeX root = ../main.tex
% -*- coding: utf-8 -*-
\chapter{总结与展望}
本章对本文的主要研究内容、创新点以及所解决的问题进行了总结,并通过对未来研究方向
的展望,提出了关于研究工作的进一步的想法和建议。

\section{研究内容总结}
软件维护是贯穿软件生命周期的重要软件活动,占据了整个软件生命周期中的大部分时间。
随着用户需求的不断添加,软件系统的规模和复杂度也不断提高,导致软件维护的难度越来
越大。与此同时,软件系统通常需要快速迭代才能适应当代社会日新月异的改变与需求,为
软件维护的效率和性能带来了重大的挑战。

本文的研究内容主要针对软件维护中最主要的两个软件维护类型--纠错性软件维护和完善性
软件维护。纠错性软件维护通过尽可能早地发现软件系统中存在的缺陷,使得软件能够正确
地运行,是完善性软件维护的基础;同时,完善性软件维护通过改进软件设计、提高软件系
统的易读性、可靠性和可维护性,能够提高纠错性软件维护的效率,并减少新缺陷的引入。

本文的研究方法为基于数据驱动的方法,通过挖掘数据中潜在的规律作为软件维护问题的解决方案。虽然目前很难
制定出完整而严密的规则来完全自动化进行软件维护,但由于软件维护的过程是存在规律的,通过数据挖掘进行软
件维护能够在提高软件维护效率和性能的同时,减少软件维护的成本。为了提高纠错性和完善性软件维护的效率和
性能,本文通过数据挖掘的方法完成了以下研究工作:

第一,为了提高纠错性软件维护的效率,本文研究了基于覆盖分析的多缺陷定位问题。由于开发过程中很难发现所
有潜在的缺陷,软件系统中存留的缺陷可能导致软件行为失效。很多研究者提出了缺陷定位方法来尽可能早地发现
导致软件行为失效的原因。其中,基于覆盖分析的缺陷定位方法由于只需要统计测试用例执行时的覆盖信息,通过
特定的公式估算代码的可疑度来定位缺陷,而不需要对源代码进行建模和分析,因此计算成本较低,在面对大规模
软件系统时适用性较强。然而,现有的基于覆盖分析的缺陷定位方法大多假设被失效用例执行越多、成功用例执行
越少的代码越可疑。这种假设虽然在单缺陷程序中可能性较大,但面对多缺陷程序时,由于造成程序失效的原因不
止一个,因此即使是缺陷相关代码也可能不被失效用例所覆盖。由于很难区分由不同缺陷造成的失效用例,导致现
有的基于覆盖分析的缺陷定位方法在面对多缺陷程序时的定位效率下降。为了提高多缺陷程序的定位效率,本文受
特征选择启发,提出了缺陷相关统计量,为每个失效用例寻找距离最近的失效用例,作为最有可能与其由相同缺陷
触发的失效用例。通过计算覆盖特征对测试用例执行结果的重要性,找到覆盖后容易导致测试用例失效的覆盖特
征,将其作为可能的缺陷相关特征,从而进行缺陷定位。


第二,为了提高完善性软件维护的效率,本文研究了函数抽取重构机会的识别和推荐。随着对软件的不断修改,软
件系统越来越偏离原来的设计,导致软件质量越来越低。为了提高软件质量,需要及时对软件系统进行重构。针对
最常用的重构类型之一,函数抽取软件重构,目前已经有很多研究者提出了自动识别和推荐重构机会的方法。大多
数函数抽取重构机会推荐方法通过软件质量度量来评价候选重构机会。虽然提高软件质量是软件重构的主要原因,
但与纠错性软件维护不同,完善性维护很难有统一的、明确的标准来量化维护的标准。除此以外,考虑到函数抽取
软件重构的原因具有多样性,根据软件质量度量进行推荐的结果往往只能得到满足某种特定软件质量度量的重构机
会,因此效果不甚理想。为了解决这个问题,本文提出通过挖掘开源软件仓库中的函数抽取重构实例,并在特征提
取算法中融合了复杂度、内聚度和耦合度三个与函数抽取重构相关的重要软件质量概念,并综合考虑多种程序元
素,本文构建了关于函数抽取重构机会的概率模型,让模型学习如何进行函数抽取重构。除此以外,通过对特征重
要性进行分析,本文研究了函数抽取重构的原因,从而进一步理解了软件维护人员的重构意图。

第三,为了提高软件系统的易读性和可维护性,本文研究了函数重命名的问题。在软件维护过程中,最耗费时间和
精力成本的是阅读代码的过程。准确的函数名可以提高代码阅读的速度,从而提高软件维护的效率;不准确的函数
名通常导致理解和维护软件系统的难度提高,甚至在某些情况下可能导致缺陷产生。尽管函数名对软件系统的易读
性和可维护性具有较大的影响,但寻找具有高度抽象能力且易读性强的函数名十分困难。随着自然语言处理技术的
发展,很多研究者将程序语言作为一种特殊语言,使用自然语言处理技术将代码当作文本来学习。尽管编程语言与
自然语言之间存在一定的相似性,但程序语言中蕴含着丰富的结构信息,直接将代码当做文本进行学习,得到的上
下文信息不够准确,因此容易导致模型效果欠佳。为了利用代码结构中所蕴含的信息,本文通过控制流分析,将函
数体拆分为由基本代码块组成的序列,每个基本代码块表示一个函数的子功能,由词项序列进行表示。利用层次注
意力机制,分别学习词项对代码块和代码块对函数体的重要性,使模型能够准确识别出对函数名预测有积极意义的
词项和代码块。通过这样的方式,模型能够学习输入代码片段的分布式表示,以及函数名在代码段上的条件概率分
布,从而为给定代码段预测函数名。

\section{创新点与主要贡献}
针对以上研究工作,本文的创新点与贡献主要有以下三个方面:

第一,本文提出了基于特征选择的缺陷定位方法,将多缺陷定位问题转化为数据挖掘中的特征选择问题。为了提高
基于覆盖分析的多缺陷定位效率,本文提出将测试用例作为样本,测试用例的执行结果作为二分类标签,测试用例
的执行覆盖信息作为样本特征,根据特征对样本类别的重要性来计算覆盖特征的可疑度。具体来说,本文借鉴了
Relief特征选择算法中的相关统计量的概念,为每个测试用例寻找距离其最近的同类和异类测试用例,距离同类越
近、距离异类越远的特征被认为越能够区分样本的类别,因此特征重要性越大。该方法能够为每个失效用例找到最
有可能与其由同一缺陷所触发的失效用例,因此有效地避免了多缺陷之间的相互影响。然而,由于根据特征选择算
法只能得到对测试用例执行结果较为重要的特征,无法得到特征的取值是如何影响测试用例的结果。因此,本文在
相关统计量的基础上提出了缺陷相关统计量,为覆盖后容易导致样本类别为失效的覆盖特征分配较高的缺陷相关统
计量,然后根据缺陷相关统计量的高低排查相关代码。最后本文为评价该方法设计了两组对比实验,分别在单缺陷
和多缺陷程序上,与较为流行的基于覆盖分析的缺陷定位方法进行比较,在开源软件程序上证明了本文方法的有效
性。

第二,本文提出通过挖掘开源软件仓库中真实存在的函数抽取重构实例,构建关于函数抽取重构的概率模型。为了
提高软件维护的效率,本文首次提出了基于机器学习的函数抽取重构推荐方法,通过对比开源软件系统中两个相邻
的版本,收集函数抽取重构实例作为训练数据集。在特征提取阶段,分别提取了结构和功能特征,融合了复杂度、
内聚度和耦合度三个与函数抽取软件重构相关的软件质量因素。与基于软件质量度量的软件重构推荐方法不同,本
文模型考虑了多种程序元素,包括变量、函数调用和类型等。在训练阶段,在将函数抽取重构实例表示为特征序列
后,使用梯度上升决策树构建并训练模型。在推荐阶段,在为给定函数体生成所有合法的候选函数抽取重构机会
后,通过训练好的模型为每个重构机会分配一个概率,按照概率由高至低推荐给用户。除此以外,本文开发了基于
Eclipse的开源插件,为软件维护人员推荐函数抽取重构机会。最后,通过在开源软件系统上的对比实验,本文比
较和分析了四种当前流行的函数抽取重构机会推荐工具的性能,并比较了梯度上升决策树与其它模型在该任务上的
优劣,实验结果证明了基于梯度上升决策树的函数抽取重构机会推荐模型的有效性。

第三,本文提出通过挖掘开源软件系统中的函数命名实例,构建关于函数命名的概率模型。为了提高软件系统的易
读性和可维护性,本文通过挖掘GitHub上广受欢迎的开源软件系统,学习函数命名的内在规律,从而为给定代码片
段推荐合适的函数名。具体来说,本文在编码-解码模型的基本框架上提出了基于层次注意力的函数名推荐模型。
在编码器中,根据代码的控制流信息,将输入代码段拆分为由多个基本代码块组成的序列,每个基本代码块为一个
词项序列;利用层次注意力机制分别学习词项对代码块、代码块对输入代码段的重要性,从而将输入代码段表示分
布式向量;在解码器中,使用基于门控循环单元的模型将编码器的输出作为解码器的输入,依次输出词项序列,直
到输出结束符号为止。在函数名推荐阶段,针对给定函数体,使用集束搜索生成具有概率的函数名列表,并按照概
率由高至低推荐给用户。最后,在10个开源软件系统上的对比实验证明了基于层次注意力的函数名推荐模型的有效
性,也证明了本文方法能够在一定程度上学到代码片段的语义。

\section{研究工作展望}
本文旨在通过数据挖掘提高软件维护关键技术的效率,针对软件维护中最主要的两种软件维护类型,纠错性和完善
性软件维护,提出了基于数据驱动的方法,将软件维护问题转化为数据挖掘问题,从数据的角度理解软件维护的过
程,构建相应的模型并实现了原型系统,从而在一定程度上提高了软件维护的效率。然而,在实际应用中,本文方
法仍然存在一定的局限性和可以改进的方向,主要体现在以下方面:

(1)基于数据驱动的方法对数据集质量的依赖性较大。由于基于数据驱动的方法通常假设能够从数据中发现潜在
的规律作为解决方案,因此数据集质量的好坏对模型的结果经常有较大的影响。如果数据集中存在较多的噪音或者
数据的分布较为不平衡,容易导致模型的性能随之降低。例如在基于特征选择的缺陷定位方法中,若测试用例集中
只存在由某一种缺陷引发的失效用例,而另一种缺陷没有或极少被触发,则很难通过该方法迅速定位到另一个缺
陷。为了避免这种情况,本文在实验中过滤了不能够被触发的软件缺陷,然而在实际使用中,这种情况很难完全避
免。再例如在基于数据驱动的软件重构推荐中,若训练数据集中夹杂着较多的噪音,则模型的表现可能会因为受到
噪音的影响而下降。


(2)虽然基于梯度上升决策树的函数抽取重构机会推荐模型,在一定程度上提高了函数抽取重构的效率,但该方
法通过特征提取的方式对函数抽取重构实例进行表示,而没有使用原始特征,因此可能导致部分与软件重构相关的
信息在特征提取阶段被丢失,从而降低函数抽取重构推荐的准确性。

(3)本文针对软件重构中最常用的两个软件重构类型进行研究,但针对其它软件重构类型,还没有通用的软件重
构机会识别和推荐模型。

针对上述本文方法的局限性,未来将进行更加深入的研究,例如:

(1)针对数据不平衡的问题,一种解决方式是建立完善的数据评价体系,通过对数据集进行抽样,根据抽样的结
果估算数据集中各类别数据的分布情况;另一个可行的方式是通过数据可视化的方式观察数据集的分布。

(2)为了尽可能地减少数据集中噪音对模型的影响,接下来一方面可以通过完善数据集收集的过程,增加数据集
的规模来减少噪音对模型的影响力;另一方面打算对模型加入一些规则化,通过降低模型的复杂度,来减少模型对
噪音数据的敏感度。

(3)关于对软件重构机会推荐的研究,本文接下来的研究方向是利用原始特征构建软件重构概率模型。虽然对重
构实例的表示是一个较大的挑战,但通过将其转化为对代码属性的预测,例如在代码表示模型的基础上,通过预测
软件重构类型在给定代码上的条件概率分布来推荐软件重构机会。

