% !TeX root = ../main.tex
% -*- coding: utf-8 -*-


\begin{zhaiyao}

当代社会,软件系统已经深入到社会生活的各个方面。在软件系统发布后,需要不断对软件进行维护,从而使软
件系统能够正确、高效地运行。其中纠错性软件维护和完善性软件维护是提高软件正确性和效率的重要手段。由于
软件维护占据整个软件生命周期的大部分,因此不能快速、可靠地进行软件维护往往给社会经济带来巨大的损失。

为了提高软件维护的效率和性能,很多研究者致力于开发自动化的软件维护工具。然而,由于软件维护自身的复杂
性,目前很难制定出完整而严密的方案来完全自动化地解决软件维护中的难题,软件维护仍然需要大量时间和人力
成本。尽管如此,软件维护的过程是存在规律的,而不是随机的、无章可循的。随着开源社区的成熟和版本控制工
具的发展,提供了越来越多的软件数据,也为基于数据驱动的软件维护带来了可能。

本文针对软件维护中最重要的两个软件维护类型,纠错性软件维护和完善性软件维护,提出了基于数据驱动的方
法,挖掘数据中的潜在规律作为软件维护问题的解决方案。从数据的角度理解软件维护的过程,将软件维护问题转
化为数据挖掘问题,能够在提高软件维护效率的同时,减少软件维护成本。

为了提高软件维护的效率和性能,本文基于数据驱动的思路做了以下研究:

(1)为了提高纠错性软件维护的效率,本文提出了基于相关统计量的缺陷定位方法,将多缺陷定位问题转化为数
据挖掘中的特征选择问题。具体而言,将每个测试用例作为样本,执行结果作为二分类标签,执行覆盖信息作为样
本特征,根据特征对样本类别的重要性计算覆盖特征的可疑度。通过为每个测试用例寻找距离其最近的同类和异类
测试用例,将距离同类越近、距离异类越远的特征作为越能够区分样本类别的特征,从而找到覆盖后容易导致测试
用例失效执行的特征作为缺陷相关覆盖特征。由于本文方法能够为每个失效用例找到最有可能与其由同一缺陷触发
的失效用例,因此能够降低多缺陷之间的相互影响,从而提高多缺陷程序的定位效率。

(2)为了提高完善性软件维护的效率,本文首次提出基于机器学习的函数抽取重构机会推荐方法。通过挖掘开源
软件仓库中真实存在的函数抽取重构实例,使用梯度上升决策树构建关于函数抽取重构的概率模型。模型中不仅融
合了复杂度、内聚度和耦合度三个重要软件质量因素,还综合考虑了变量、类型、函数调用等多种程序元素。在推
荐阶段,本文基于``生成并排序''的方法,首先为给定函数生成所有合法的函数抽取重构机会,然后通过模型为每
个重构机会分配一个概率,按照概率由高至低推荐给用户。本文设计并开发了基于Eclipse的函数抽取重构推荐工
具,通过为用户推荐带有概率的函数抽取重构机会,提高了软件维护的效率。

(3)为了提高软件系统的易读性和可维护性,本文在编码-解码模型的框架上提出了基于层次注意力的函数名推荐
方法。通过挖掘开源软件系统,学习函数命名的内在规律,从而为给定代码片段推荐能够抽象代码功能的函数名。
具体而言,在编码器中,本文根据代码的控制流信息,将输入代码段拆分为由多个基本代码块组成的序列,每个代
码基本块为一个词项序列;利用层次注意力机制分别学习词项对代码块、代码块对代码片段的重要性,从而将输入
代码段编码为一个分布式向量;在解码器中,使用基于门控循环单元的模型依次预测词项,直到输出结束符号为
止。本文通过学习函数名在代码片段的分布式表示上的条件概率,为代码片推荐具有概括抽象能力的函数名,从而
提高软件系统的易读性、易理解性和可维护性,在一定程度上提高了软件维护的效率。
\end{zhaiyao}

\begin{guanjianci}
	数据驱动;软件维护;软件质量度量
\end{guanjianci}

\begin{abstract}

With the development of information technology, software systems have been applied to almost all
aspects of modern society. Once software system is released, it has to be maintained continuously,
so that the software system can run correctly and efficiently. Corrective and perfective maintenance
are major types of maintenance that can improve the correctness and efficiency of software systems.
For the reason that software maintenance occupies the majority of the entire software life cycle,
software systems that cannot be maintenanced efficiently and reliably can cause huge financial
losses to society.

To improve the efficiency and effectiveness of software maintenance, much research has been done to
develop automated software maintenance tools. However, due to the complexity of software
maintenance, it is currently difficult to develope completely automated tools to solve the problems.
Software maintenance still requires a large amount of time and labor. Despite this, there exists
some rules that can be mined from the process of software maintenance. As the open source community
matures, more and more software data has been available, which establishes the foundation of
data-driven software maintenance.

This dissertation focuses on two major types of software maintenance, i.e., corrective and
perfective maintenance, and proposes data-driven approaches to solve critical problems in software
maintenance. In order to improve the efficiency and performance of software maintenance, this
dissertation conducts the following research:

(1) In order to improve the efficiency of corrective maintenance, this dissertation proposes a
defect localization approach that transforms the problem of multi-defect localization into the a
feature selection problem. Specifically, by taking each test case as a sample, the coverage spectra
of it can be considered as the features, and whether the test case passes or fails is naturally the
label. Suspiciousness of each feature can be calculated according to the importance of it. Inspired
by the Relief algorithm, this dissertation proposes to find features that can distinguish between
the positive and negative samples effectively. Features that are likely to leads to the failures can
be identified as defect-prone features. Since the proposed method can find the failure that is most
likely to be triggered by the same defect for each failure, the influence between multiple defects
can be reduced, and the proposed method can improve the efficiency of defect localization.

(2) In order to improve the efficiency of perfective maintenance, this dissertation proposes a
machine learning-based method to recommend Extract Method refactoring opportunities. By mining
real-world refactorings in open source software, this dissertation builds a probabilistic model of
Extract Method refactorings based on gradient boosting decision tree (GBDT). During feature
extraction, the proposed model not only incorporates the idea of complexity, cohesion and coupling,
but also considers various program elements such as variables, types and method invocations. This
model is a ``generate-and-rank'' system, which first generates all valid Extract Method refactoring
opportunities for a given method, and then assigns a probability to each refactoring opportunity, by
which it recommends Extract Method refactoring opportunities to users. Moreover, this dissertation
designs and implements an Eclipse-plugin to recommend Extract Method refactorings, so as to improve
the efficiency of software maintenance.

(3) In order to improve the readability and maintainability of software systems, this dissertation proposes
a deep learning model that explores hierarchical attention mechanisms to suggest names for methods.
This model is built based on the encoder-decoder framework. By mining open source software, the
proposed model can learn the inherent rules of naming methods. Specifically, in the encoder, given a
piece of code fragment, the model first splits it into a sequence of basic blocks, where each basic
block is a sequence of tokens. Through hierarchical attention mechanisms, the model can learn the
importance of each token and that of each basic block respectively. In this way, the model can
represent the code fragment by a distributed vector. In the decoder, based on the Gate Recurrent
Unit (GRU) model, the proposed model predicts the tokens that can form a method name sequentially.
By learning the conditional probability distribution of method names over code fragments, this model
can suggest names for methods effectively, in order to improve the readability, comprehensibility
and reliability of software systems.

\end{abstract}



\begin{keywords}
	Data Driven; Software Maintenance; Software Quality Metrics
\end{keywords}