% !TeX root = ../main.tex
% -*- coding: utf-8 -*-

\chapter{基于相关统计量的缺陷定位}
具体来说,Relief为每个测试用例寻找距离最近的同类和异类样本,距离同类越近、距离异类越远的特征被认为越能够区分不同类别的样本。
\section{引言}
\LaTeX 提供宏命令\verb+\frac+, 用以打印分数. 为使得版面整齐, 该命令的使用应遵循以下原则:

\begin{enumerate}
\item 仅在分行表达式中使用,
\item 不嵌套使用,
\item 不在上下标中使用.
\end{enumerate}

也就是说, 行内表达式和上下标中出现分数时一律用 $a/b$表示, 如
$(x+2)/((3x^2+4)(7+y))$. 下面是居中表达式:

\[
 x^2 = y^{1/2} +3.
\]

多行表达式: 尽量在加、减、乘、等号前换行. 在乘号前换行时,
下一行首用 \string\times:
\def\iint{\mathop{\int\!\!\!\int}}\def\calG{\mathcal G}
\begin{eqnarray}
&&\left|(W_{\psi_1}f)(a,b)-(W_{\psi_1}f)(a_j,b_{j,k})\right|^{2}\nonumber\\
&=&\frac{1}{C^{2}_{\varphi}}\Bigg|\iint_{\calG} (W_{\varphi}f)(s,t) \nonumber\\
&&\qquad\times \Bigg( (W_{\psi_1}\varphi)\left(\frac{a}{s},
\frac{b-t}{s}\right)
     -(W_{\psi_1}\varphi)\left(\frac{a_{j}}{s}, \frac{b_{j,k}-t}{s}\right)\Bigg)
  \frac{dsdt}{s^{d+1}}\Bigg|^2 \nonumber\\
&\le& \frac{1}{C^2_{\varphi}} \iint_{\calG} |(W_{\varphi}f)(s,t)|^2 \nonumber\\
&&\qquad \times\left| (W_{\psi_1}\varphi)\left(\frac{a}{s},
\frac{b-t}{s}\right)
    -(W_{\psi_1}\varphi)\left(\frac{a_{j}}{s}, \frac{b_{j,k}-t}{s}\right)\right|
   \frac{dsdt}{s^{d+1}}  \nonumber\\
&&\qquad \times   \iint_{\calG}\!
 \left|(W_{\psi_1}\varphi)\left(\frac{a}{s}, \frac{b-t}{s}\right)
    -(W_{\psi_1}\varphi)\left(\frac{a_{j}}{s}, \frac{b_{j,k}-t}{s}\right)\right|
 \frac{ ds dt}{s^{d+1}} \nonumber\\
&=& \frac{1}{C^2_{\varphi}} ....  \label{eq:a0}
\end{eqnarray}


科技文献中一般用半角标点,,, 请参考《中国科学》发表的论文.

如果使用全角标点, 可以使用
\begin{verbatim}
  \punctstyle{<style>}
\end{verbatim}
选择标点样式, 有效值为
\begin{verbatim}
  quanjiao (所有标点符号占一个汉字宽度,
            相邻两个标点占一个半汉字宽度)
  banjiao  (所有标点符号占半个汉字宽度)
  hangmobanjiao (所有标点符号占一个汉字宽度,行末行首半角)
  kaiming  (句号、叹号、问号占一个汉字宽度,其他标点占半个汉字宽度)
\end{verbatim}
缺省为全角式。注意:不论选择哪种样式,都提供行末对齐(margin kerning)功能。



\begin{Theorem} \label{thm:latex}
\LaTeX 的输出是最完美的.
\end{Theorem}

先证明一个引理
\begin{Lemma} \label{thm:tex}
\TeX 文件在不同操作系统下的排版结果完全一致.
\end{Lemma}

\begin{proof}
这是证明.
\end{proof}
\section{相关统计量介绍}
\section{研究方法}
\subsection{分支覆盖特征谱}
\subsection{基于相关统计量的缺陷定位方法}
\section{实验设置}
\subsection{数据集和对比实验}
\subsection{评价指标}
\section{实验结果}
\section{讨论}
\section{本章小结}
